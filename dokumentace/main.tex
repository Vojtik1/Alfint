\documentclass[12pt, a4paper]{report}

\usepackage[T1]{fontenc}
\usepackage[utf8]{inputenc}
\usepackage[czech]{babel}
\usepackage{graphicx}
\usepackage{lmodern}
\usepackage{tikz}

%% Proměnné
\newcommand\obor{INFORMAČNÍ TECHNOLOGIE}
\newcommand\kodOboru{18-20-M/01}
\newcommand\zamereni{se zaměřením na počítačové sítě a programování}
\newcommand\skola{Střední škola průmyslová a umělecká, Opava}
\newcommand\trida{IT4}
\newcommand\jmenoAutora{Vojtěch Šíma}
\newcommand\skolniRok{2024/25}
\newcommand\datumOdevzdani{31. 12. 2024}
\newcommand\nazevPrace{Alfint - Aplikace pro zpětné testování akcií}

\title{\nazevPrace}
\author{\jmenoAutora}
\date{\datumOdevzdani}

\usepackage[top=2.5cm, bottom=2.5cm, left=3.5cm, right=1.5cm]{geometry} %% nastaví okraje, left -- vnitřní okraj, right -- vnější okraj

\usepackage[czech]{babel} %% balík babel pro sazbu v češtině
\usepackage[utf8]{inputenc} %% balíky pro kódování textu
\usepackage[T1]{fontenc}
\usepackage{cmap} %% balíček zajišťující, že vytvořené PDF bude prohledávatelné a kopírovatelné
\usepackage{lmodern}  % Přidáno pro lepší podporu českých znaků

\usepackage{graphicx} %% balík pro vkládání obrázků

\usepackage{subcaption} %% balíček pro vkládání podobrázků

\usepackage{hyperref} %% balíček, který v PDF vytváří odkazy

\linespread{1.25} %% řádkování
\setlength{\parskip}{0.5em} %% odsazení mezi odstavci


\usepackage[pagestyles]{titlesec} %% balíček pro úpravu stylu kapitol a sekcí

% Nastavení formátování nadpisů
\titleformat{\chapter}[hang]
  {\rmfamily\bfseries\Large}  % Zmenšeno z \LARGE na \Large
  {\thechapter}
  {1em}
  {\MakeUppercase}

\titleformat{\section}[hang]
  {\rmfamily\bfseries\large}  % Zmenšeno z \Large na \large
  {\thesection}
  {1em}
  {}

\titleformat{\subsection}[hang]
  {\rmfamily\bfseries\normalsize}  % Zmenšeno z \large na \normalsize
  {\thesubsection}
  {1em}
  {}

% Nastavení mezer před a za nadpisy
\titlespacing*{\chapter}{0pt}{-30pt}{8pt}  % Zmenšeno z 20pt na 8pt
\titlespacing*{\section}{0pt}{12pt}{4pt}   % Zmenšeno z 20pt/10pt na 12pt/4pt
\titlespacing*{\subsection}{0pt}{12pt}{4pt} % Zmenšeno z 20pt/10pt na 12pt/4pt

% Nastavení formátování pro nečíslované nadpisy
\titleformat{name=\chapter,numberless}[hang]
  {\rmfamily\bfseries\Large}
  {}
  {0pt}
  {\MakeUppercase}

\titleformat{name=\section,numberless}[hang]
  {\rmfamily\bfseries\large}  % Zmenšeno z \Large na \large
  {}
  {0pt}
  {}

% Nastavení mezer pro nečíslované nadpisy
\titlespacing*{name=\chapter,numberless}{0pt}{0pt}{8pt}  % Změněno z -30pt na 0pt pro první mezeru
\titlespacing*{name=\section,numberless}{0pt}{12pt}{4pt}   % Zmenšeno z 15pt/5pt na 12pt/4pt

% Nastavení fontu pro obsah
\renewcommand{\cfttoctitlefont}{\rmfamily\LARGE\bfseries}
\renewcommand{\cftchapfont}{\rmfamily\bfseries}
\renewcommand{\cftsecfont}{\rmfamily}
\renewcommand{\cftsubsecfont}{\rmfamily}

% Nastavení číslování pro obsah
\renewcommand{\cftchappresnum}{\rmfamily\bfseries}
\renewcommand{\cftsecpresnum}{\rmfamily}
\renewcommand{\cftsubsecpresnum}{\rmfamily}

\usepackage{tocloft} % Balíček pro přizpůsobení vzhledu obsahu
\setlength{\cftbeforechapskip}{10pt}  % Větší rozestup pro kapitoly
\setlength{\cftbeforesecskip}{3pt}   % Menší rozestup pro sekce

% Nastavení odsazení a formátování obsahu
\cftsetindents{chapter}{0em}{2.5em}
\cftsetindents{section}{2.5em}{3em}
\cftsetindents{subsection}{5.5em}{3.7em}
\renewcommand{\cftdotsep}{2} % Hustota teček
\renewcommand{\cftchapleader}{\cftdotfill{\cftdotsep}} % Tečky i pro kapitoly

\setcounter{secnumdepth}{2}
\setcounter{tocdepth}{2}
\usepackage{fancyhdr}
\pagestyle{fancy}
\renewcommand{\headrulewidth}{0.025pt}

\usepackage{booktabs}

\usepackage{url}

%% Balíčky co se můžou hodit :) 
%%%%%%%%%%%%%%%%%%%%%%%%%%%%%%%

\usepackage{pdfpages} %% Balíček umožňující vkládat stránky z PDF souborů, 

\usepackage{upgreek} %% Balíček pro sazbu stojatých řeckých písmen, třeba u jednotky mikrometr. Například stojaté mí: \upmu, stojaté pí: \uppi

\usepackage{amsmath}    %% Balíčky amsmath a amsfonts 
\usepackage{amsfonts}   %% pro sazbu matematických symbolů
\usepackage{esint}     %% pro sazbu různých integrálů (např \oiint)
\usepackage{mathrsfs}
\usepackage{helvet} % Helvet font
\usepackage{mathptmx} % Times New Roman
\usepackage{Oswald} % Oswald font


%% makra pro sazbu matematiky
\newcommand{\dif}{\mathrm{d}} %% makro pro sazbu diferenciálu, místo toho
%% abych musel psát '\mathrm{d}' mi stačí napsat '\dif' což je mnohem 
%% kratší a mohu si tak usnadnit práci

\usepackage{listings}
\usepackage{xcolor}

\renewcommand{\lstlistingname}{Kód}% Listing -> Algorithm
\renewcommand{\lstlistlistingname}{Seznam programových kódů}% List of Listings -> List of Algorithms

%% Definice 
\lstdefinelanguage{JavaScript}{
	morekeywords=[1]{break, continue, delete, else, for, function, if, in,
		new, return, this, typeof, var, void, while, with},
	% Literals, primitive types, and reference types.
	morekeywords=[2]{false, null, true, boolean, number, undefined,
		Array, Boolean, Date, Math, Number, String, Object},
	% Built-ins.
	morekeywords=[3]{eval, parseInt, parseFloat, escape, unescape},
	sensitive,
	morecomment=[s]{/*}{*/},
	morecomment=[l]//,
	morecomment=[s]{/**}{*/}, % JavaDoc style comments
	morestring=[b]',
	morestring=[b]"
}[keywords, comments, strings]


\lstdefinelanguage[ECMAScript2015]{JavaScript}[]{JavaScript}{
	morekeywords=[1]{await, async, case, catch, class, const, default, do,
		enum, export, extends, finally, from, implements, import, instanceof,
		let, static, super, switch, throw, try},
	morestring=[b]` % Interpolation strings.
}

\lstalias[]{ES6}[ECMAScript2015]{JavaScript}

% Nastavení barev
% Requires package: color.
\definecolor{mediumgray}{rgb}{0.3, 0.4, 0.4}
\definecolor{mediumblue}{rgb}{0.0, 0.0, 0.8}
\definecolor{forestgreen}{rgb}{0.13, 0.55, 0.13}
\definecolor{darkviolet}{rgb}{0.58, 0.0, 0.83}
\definecolor{royalblue}{rgb}{0.25, 0.41, 0.88}
\definecolor{crimson}{rgb}{0.86, 0.8, 0.24}

% Nastavení pro Python
\lstdefinestyle{Python}{
	language=Python,
	backgroundcolor=\color{white},
	basicstyle=\ttfamily,
	breakatwhitespace=false,
	breaklines=false,
	captionpos=b,
	columns=fullflexible,
	commentstyle=\color{mediumgray}\upshape,
	emph={},
	emphstyle=\color{crimson},
	extendedchars=true,  % requires inputenc
	fontadjust=true,
	frame=single,
	identifierstyle=\color{black},
	keepspaces=true,
	keywordstyle=\color{mediumblue},
	keywordstyle={[2]\color{darkviolet}},
	keywordstyle={[3]\color{royalblue}},
	literate=%
	{á}{{\'a}}1 {č}{{\v{c}}}1 {ď}{{\v{d}}}1 {é}{{\'e}}1 {ě}{{\v{e}}}1
	{í}{{\'i}}1 {ň}{{\v{n}}}1 {ó}{{\'o}}1 {ř}{{\v{r}}}1 {š}{{\v{s}}}1
	{ť}{{\v{t}}}1 {ú}{{\'u}}1 {ů}{{\r{u}}}1 {ý}{{\'y}}1 {ž}{{\v{z}}}1,		
	numbers=left,
	numbersep=5pt,
	numberstyle=\tiny\color{black},
	rulecolor=\color{black},
	showlines=true,
	showspaces=false,
	showstringspaces=false,
	showtabs=false,
	stringstyle=\color{forestgreen},
	tabsize=2,
	title=\lstname,
	upquote=true  % requires textcomp	
}


\lstdefinestyle{JSES6Base}{
	backgroundcolor=\color{white},
	basicstyle=\ttfamily,
	breakatwhitespace=false,
	breaklines=false,
	captionpos=b,
	columns=fullflexible,
	commentstyle=\color{mediumgray}\upshape,
	emph={},
	emphstyle=\color{crimson},
	extendedchars=true,  % requires inputenc
	fontadjust=true,
	frame=single,
	identifierstyle=\color{black},
	keepspaces=true,
	keywordstyle=\color{mediumblue},
	keywordstyle={[2]\color{darkviolet}},
	keywordstyle={[3]\color{royalblue}},
 literate=%
{á}{{\'a}}1 {č}{{\v{c}}}1 {ď}{{\v{d}}}1 {é}{{\'e}}1 {ě}{{\v{e}}}1
{í}{{\'i}}1 {ň}{{\v{n}}}1 {ó}{{\'o}}1 {ř}{{\v{r}}}1 {š}{{\v{s}}}1
{ť}{{\v{t}}}1 {ú}{{\'u}}1 {ů}{{\r{u}}}1 {ý}{{\'y}}1 {ž}{{\v{z}}}1,		
	numbers=left,
	numbersep=5pt,
	numberstyle=\tiny\color{black},
	rulecolor=\color{black},
	showlines=true,
	showspaces=false,
	showstringspaces=false,
	showtabs=false,
	stringstyle=\color{forestgreen},
	tabsize=2,
	title=\lstname,
	upquote=true  % requires textcomp
}

\lstdefinestyle{JavaScript}{
	language=JavaScript,
	style=JSES6Base,
}
\lstdefinestyle{ES6}{
	language=ES6,
	style=JSES6Base
}


%% Bordel pro práci - můžeš smáznout :) 
%%%%%%%%%%%%%%%%%%%

\usepackage{lipsum} %% balíček který píše lipsum (nesmyslný text, který se používá pro kontrolu typografie)

\begin{document}
	
	\pagestyle{empty}
	\pagenumbering{arabic}
	
	\cleardoublepage
%% Titulní stránka s informacemi
%%%%%%%%%%%%%%%%%%%%%%%%%%%%%%%%%%%%%%%%
	
	{\fontfamily{phv}\selectfont
		%% Logo školy
		\begin{figure}[h]
			\centering
			\includegraphics[width=0.6\linewidth]{logo-skola.png}
		\end{figure}
		
		
		%% Hlavička práce a její název (viz proměnná \nazev prace)
		%% \sffamily %%% bezpatkové písmo - sans serif
		{\bfseries %%% písmo na stránce je tučně
			\begin{center}
				\vspace{0.025 \textheight}
				\LARGE{ZÁVĚREČNÁ STUDIJNÍ PRÁCE}\\
				\large{dokumentace}\\
				\vspace{0.075 \textheight}
				\LARGE {\nazevPrace}\\
			\end{center}  
		}%%%

        \begin{figure}[h]
			\centering
			\includegraphics[width=0.8\linewidth]{ndAI.png}
			\label{fig:main-screen}
		\end{figure}

    \vspace{0.02 \textheight}
		\begin{table}[h!]
			\begin{tabular}{ll}
				\textbf{Autor:} & \jmenoAutora\\ 
				\textbf{Obor:} & \kodOboru { } \obor\\
				\textbf{} & \zamereni\\
				\textbf{Třída:} & \trida\\
				\textbf{Školní rok:} & \skolniRok\\
			\end{tabular}
			
		\end{table}		
	}
	
	\clearpage

    
%% Stránka obsahující poděkování a prohlášení
%%%%%%%%%%%%%%%%%%%%%%%%%%%%%%%%%%%%%%%%%%%%%%%%%%%%%%%%

%% Poděkování - nepovinné
%%%%%%%%%%%%%%%%%%%%%%%%%%%%
	
	\noindent{\large{\bfseries{Poděkování}\\}}
	\noindent Děkuji všem, co mě psychicky podpořili, když jsem v kódu měl tisíce errorů.
	
	\vspace*{0.7\textheight} %% Vertikální mezeru je možné upravit

%% Prohlášení - povinné
%%%%%%%%%%%%%%%%%%%%%%%%%%%%
	\noindent{\large{\bfseries{Prohlášení}\\}}  %% uprav si koncovky podle toho na jaký rod se cítíš, vypadá to pak lépe :) 
	\noindent{Prohlašuji, že jsem závěrečnou práci vypracoval samostatně a uvedl veškeré použité 
		informační zdroje.\\}
	\noindent{Souhlasím, aby tato studijní práce byla použita k výukovým a prezentačním účelům na Střední průmyslové a umělecké škole v Opavě, Praskova 399/8.}
	\vfill
	\noindent{V Opavě \datumOdevzdani\\}
	\noindent
	\begin{minipage}{\linewidth}
		\hspace{9.5cm} 
		\begin{tabular}{@{}p{6cm}@{}}
			\dotfill \\
			Podpis autora
		\end{tabular}
	\end{minipage}
	
	\clearpage

	\tableofcontents
	\clearpage
%%%%%%%%%%%%%%%%%%%%%%%%%%%%%%%%%%%%%%%%%%%%%%%%%%%%%%%%	
%% Stránka obsahující abstrakt (anotaci)
%%%%%%%%%%%%%%%%%%%%%%%%%%%%%%%%%%%%%%%%%%%%%%%%%%%%%%%%	
\noindent{\Large{\bfseries{Abstrakt}\\}}
Tato práce se zaměřuje na vývoj aplikace pro zpětné testování akcií na základě uživatelsky definovaných parametrů. Aplikace je implementována v jazyce Python s využitím frameworku Django a API SimFin a yfinance. Umožňuje analyzovat historická data, sdílet portfolia a vizualizovat je v přehledné podobě. Dokumentace popisuje architekturu aplikace, implementační detaily a způsoby řešení. Důraz je kladen na přesnost výpočtů a uživatelskou přívětivost. 

\vspace{18pt}
\noindent{\large{\bfseries{Klíčová slova}}}

\noindent Python, Django, SimFin, yfinance, zpětné testování, finanční analýza, akcie


%%%%%%%%%%%%%%%%%%%%%%%%%%%%%%%%%%%%%%%%%%%%%%%%%%%%%%%%	
%% Seznam použitých zkratek
%%%%%%%%%%%%%%%%%%%%%%%%%%%%%%%%%%%%%%%%%%%%%%%%%%%%%%%%	
\chapter*{Seznam použitých zkratek}
\begin{tabular}{ll}
API & Application Programming Interface,\\
EBITDA & Earnings Before Interest, Taxes, Depreciation, and Amortization,\\
EPS & Earnings Per Share,\\
P/E & Price-to-Earnings Ratio,\\
ROE & Return on Equity,\\
REST & Representational State Transfer,\\
SQL & Structured Query Language,\\
UI & User Interface.\\
\end{tabular}

%% Definice příkazu pro podnadpisy v úvodu
\newcommand{\introsubheading}[1]{%
  {\noindent\textbf{\normalsize #1}\vspace{1pt}\par}%
}

\chapter*{ÚVOD}
\addcontentsline{toc}{chapter}{Úvod}

\introsubheading{Představení projektu}
V současné době, kdy roste zájem o investování a analýzu akciového trhu, je zpětné testování strategií nezbytným nástrojem pro investory. Tento projekt byl vytvořen s cílem poskytnout uživatelům snadno ovladatelnou aplikaci, která umožňuje analyzovat historická data akcií, vypočítávat finanční indikátory a vytvářet přehledné vizualizace. Aplikace je implementována v Pythonu s využitím frameworku Django, API SimFin a yfinance.

\introsubheading{Motivace}
Hlavní motivací pro vytvoření této aplikace byla osobní zkušenost s nedostatky dostupných nástrojů pro zpětné testování akcií. Většina stávajících řešení postrádá možnost kombinace dat z různých zdrojů a vizualizace klíčových finančních ukazatelů. Cílem bylo vytvořit nástroj, který by nabídl komplexní analýzu a zároveň byl uživatelsky přívětivý.

\introsubheading{Cíle projektu}
Hlavním cílem bylo vytvořit aplikaci, která umožní:
\begin{itemize}
    \item Analyzovat historická data akcií a finančních ukazatelů,
    \item Vypočítávat klíčové indikátory, jako jsou P/E, ROE, EPS a další,
    \item Vizualizovat data prostřednictvím grafů a přehledných tabulek,
    \item Kombinovat data z více zdrojů (SimFin, yfinance),
    \item Poskytnout uživatelům nástroj pro testování investičních strategií,
    \item Sdílet mezi sebou portfolia.
\end{itemize}

\introsubheading{Struktura práce}
Práce je rozdělena do několika hlavních částí. První kapitola se zabývá teoretickou částí, která popisuje klíčové finanční ukazatele a jejich význam při zpětném testování akcií. Druhá kapitola se věnuje návrhu aplikace, včetně použité architektury, databázového modelu a výběru technologií. Třetí kapitola popisuje implementační část, kde jsou detailně rozebrány funkce aplikace, integrace API a výpočet ukazatelů. Čtvrtá kapitola shrnuje dosažené výsledky a uzavírá práci pohledem na budoucí rozvoj projektu.

\chapter{Teoretická část}


\section{Principy zpětného testování akcií}


Zpětné testování (backtesting) je klíčovým nástrojem v oblasti finančního investování, který umožňuje ověřit účinnost různých investičních strategií na historických datech. Tento proces poskytuje cenné informace o tom, jak by se strategie chovala v reálných tržních podmínkách, pokud by byla aplikována v minulosti. Základním principem zpětného testování je analýza historických cen akcií, ziskovosti a dalších ukazatelů, které umožňují modelovat chování akcií a simulovat různé obchodní scénáře.

\section{Finanční ukazatele}

Finanční ukazatele jsou zásadní pro analýzu výkonnosti společnosti a rozhodování o investicích. V rámci zpětného testování se zaměřujeme na ukazatele, které poskytují přehled o hodnotě, stabilitě a růstovém potenciálu společnosti. Některé z nejdůležitějších ukazatelů zahrnují:

\begin{itemize}
    \item \textbf{P/E (Price-to-Earnings Ratio)} – Poměr ceny akcie k jejímu zisku na akcii. Tento ukazatel je často používán k určení, zda je akcie přiměřeně oceněná na základě jejího zisku. Vyšší P/E může znamenat, že investoři očekávají vysoký růst společnosti, zatímco nízké P/E může naznačovat podhodnocení.
    
    \item \textbf{ROE (Return on Equity)} – Tento ukazatel ukazuje, jak efektivně společnost využívá kapitál investovaný jejími akcionáři. Vyšší ROE obvykle signalizuje silnou výkonnost společnosti, což může být pozitivním signálem pro investory.

    \item \textbf{EPS (Earnings per Share)} – Zisk na akcii je jedním z nejzákladnějších ukazatelů výkonnosti společnosti. Ukazuje, kolik zisku společnost vydělala na jednu akcii. Tento ukazatel je důležitý pro hodnocení profitability společnosti.

    \item \textbf{P/B (Price-to-Book Ratio)} – Poměr ceny akcie k její účetní hodnotě. Tento ukazatel se používá pro analýzu toho, jak je akcie oceněná ve vztahu k jejím aktivům. Poměr menší než 1 může naznačovat podhodnocení akcie.

    \item \textbf{PEG (Price-to-Earnings Growth Ratio)} – Tento ukazatel kombinuje P/E s růstem zisku. Pomáhá určit, zda je akcie přiměřeně oceněná na základě očekávaného růstu zisku, což poskytuje vyváženější pohled než samotný P/E.

    \item \textbf{Debt-to-Equity Ratio (D/E)} – Poměr dluhu k vlastnímu kapitálu ukazuje, jak společnost financuje své aktivity. Vyšší poměr znamená větší závislost na cizích zdrojích financování, což může znamenat vyšší riziko.
\end{itemize}

\section{Metoda zpětného testování}


Zpětné testování se provádí tak, že se na historických datech testuje vybraná investiční strategie. Proces zahrnuje následující kroky:
\begin{itemize}
    \item \textbf{Výběr historických dat} – Získání historických cen akcií a dalších relevantních finančních dat pro dané období.
    \item \textbf{Implementace strategie} – Definování a aplikování obchodní strategie na základě vybraných ukazatelů (např. P/E, ROE).
    \item \textbf{Simulace obchodních rozhodnutí} – Vytváření simulace, která na základě historických dat a strategie simuluje nákup a prodeje akcií.
    \item \textbf{Vyhodnocení výsledků} – Analýza výsledků zpětného testování, zhodnocení výnosů a rizik spojených se strategií.
\end{itemize}

\section{Historie zpětného testování}


Historie zpětného testování sahá až do 70. let 20. století, kdy byly první pokusy o testování obchodních strategií na historických datech prováděny manuálně. S rozvojem výpočetní techniky se zpětné testování stalo mnohem dostupnější a přesnější. V současnosti se používají sofistikované algoritmy a programy, které umožňují rychlé a efektivní testování strategií na velkých historických datech.

\section{Závěr}


Zpětné testování je neocenitelný nástroj pro každého investora, který chce ověřit efektivitu své obchodní strategie na historických datech. Kombinace správných finančních ukazatelů, analýzy historických dat a využití pokročilých technik zpětného testování pomáhá minimalizovat rizika a maximalizovat potenciální výnosy.

\chapter{Návrh řešení}


\section{Architektura aplikace}


Architektura aplikace byla navržena s cílem zajistit modularitu, škálovatelnost a efektivní správu dat. Aplikace je postavena na frameworku Django, který je doplněn knihovnami jako yfinance pro získávání aktuálních a historických dat a SimFin pro přístup k finančním ukazatelům. Celková architektura aplikace zahrnuje následující komponenty:

\begin{itemize}
    \item \textbf{Frontend} – Uživatelské rozhraní aplikace je navrženo tak, aby bylo přehledné a intuitivní. Umožňuje uživatelům zadávat parametry pro zpětné testování a zobrazuje výsledky ve formě tabulek a grafů.
    \item \textbf{Backend} – Backend je zodpovědný za získávání dat z externích API (SimFin a yfinance), výpočty finančních ukazatelů a správu databáze. Django REST framework je použit pro vytvoření API pro komunikaci mezi frontendem a backendem.
    \item \textbf{Databáze} – Databáze je navržena tak, aby uchovávala historická data o akciích, obchodní strategie a výsledky zpětného testování. Používá se PostgreSQL pro její robustnost a škálovatelnost.
    \item \textbf{Grafy a vizualizace} – Pro vizualizaci výsledků zpětného testování jsou použity knihovny jako Matplotlib a Plotly. Grafy umožňují uživatelům snadno analyzovat výsledky obchodních strategií.
\end{itemize}

\section{Použité technologie}


Aplikace využívá několik moderních technologií pro efektivní získávání dat, jejich analýzu a vizualizaci. Mezi hlavní technologie použité v aplikaci patří:

\begin{itemize}
    \item \textbf{Django} – Framework pro vývoj webových aplikací, který umožňuje rychlý vývoj a snadnou správu backendu aplikace. Django poskytuje robustní strukturu pro zpracování požadavků, správu databáze a zabezpečení aplikace.
    \item \textbf{yfinance} – Knihovna pro získávání historických a aktuálních dat o akciích z Yahoo Finance. Tato knihovna je klíčová pro získání historických cen akcií a dalších relevantních dat pro zpětné testování.
    \item \textbf{SimFin} – API pro získávání finančních ukazatelů a fundamentálních dat o akciích. SimFin poskytuje přístup k datům, které jsou užitečné pro výpočet ukazatelů jako P/E, ROE, EPS a dalších.
    \item \textbf{PostgreSQL} – Relational database management system, který je použit pro správu dat o akciích, uživatelských strategiích a výsledcích testování. PostgreSQL byla zvolena pro svou spolehlivost, výkon a schopnost pracovat s velkými objemy dat.
    \item \textbf{Matplotlib a Plotly} – Knihovny pro tvorbu grafů a vizualizací, které jsou použity pro zobrazení výsledků zpětného testování a historických dat.
\end{itemize}

\section{Databázový model}


Databázový model aplikace je navržen tak, aby efektivně uchovával všechna potřebná data pro zpětné testování akcií a správu uživatelských portfolií. Model obsahuje několik hlavních tabulek, které uchovávají informace o akciích, jejich finančních ukazatelích a dalších relevantních údajích.

\begin{figure}[h]
			\centering
			\includegraphics[width=1\linewidth]{0hCv.png}
		\end{figure}

Tento databázový model umožňuje efektivní správu a analýzu finančních dat o akciích a uživatelských portfoliích, což je klíčové pro funkčnost aplikace pro zpětné testování akcií.



\section{Bezpečnostní a výkonnostní aspekty}

Aplikace je navržena s ohledem na bezpečnost uživatelských dat a ochranu proti běžným bezpečnostním hrozbám. Využívá bezpečnostní mechanismy, které poskytuje framework Django, včetně autentizace uživatelů, ochrany před CSRF útoky a dalších opatření.

\begin{itemize}
    \item \textbf{Autentizace a autorizace} – Aplikace využívá vestavěnou autentizaci Django pro správu přihlašování uživatelů. Při přihlašování uživatel zadá své uživatelské jméno a heslo, které jsou ověřeny proti databázi uživatelů. Django automaticky šifruje hesla pomocí algoritmu bcrypt, což zajišťuje jejich bezpečné uchování. Pro ochranu přihlášení je použita funkce \texttt{login\_required}, která zajišťuje, že pouze přihlášení uživatelé mohou přistupovat k určitým částem aplikace, jako jsou portfolia uživatelů.
    
    \item \textbf{CSRF ochrana} – Pro ochranu proti CSRF (Cross-Site Request Forgery) útokům Django automaticky přidává tokeny do každého formuláře. Tento token musí být odeslán spolu s každým požadavkem, což brání neautorizovanému zasílání požadavků na server. Formuláře v aplikaci obsahují příkaz \texttt{\{ \% csrf\_token \% \}} pro generování a ověřování tokenů.
    
    \item \textbf{Šifrování dat} – Django automaticky šifruje hesla uživatelů a používá HTTPS pro šifrování komunikace mezi serverem a klientem, což zajišťuje bezpečnost přenášených dat.
    
    \item \textbf{Oprávnění a přístupová kontrola} – V aplikaci je implementována kontrola přístupu na základě uživatelských oprávnění. Příkladem je použití dekorátoru \texttt{@login\_required}, který zajistí, že pouze přihlášení uživatelé mohou provádět určité akce, jako je změna stavu sdílení portfolia:
    \begin{lstlisting}[style=Python, caption= Login required] 
@login_required
def toggle_share(request, portfolio\_id):
portfolio = get_object_or_404(Portfolio, user=request.user)
portfolio.is_shared = not portfolio.is_shared
portfolio.save()
return redirect('view_portfolio', portfolio_id=portfolio_id)
    \end{lstlisting}
    Tento dekorátor zajistí, že pouze uživatel, který vlastní konkrétní portfolio, může změnit jeho stav (např. sdílení portfolia).
    
    \item \textbf{Další bezpečnostní opatření} – Aplikace využívá Django zabezpečené mechanismy, jako je ochrana proti SQL injection a XSS (Cross-Site Scripting). Všechny formuláře jsou ošetřeny proti těmto útokům, což zajišťuje bezpečnost aplikace.
\end{itemize}

Výkonnost aplikace je také optimalizována pro efektivní zpracování velkých objemů dat. Použití asynchronního zpracování a cachování výsledků zaručuje, že aplikace bude reagovat rychle i při vysokém zatížení.

\chapter{Implementační část}


\section{Získávání dat z externích API}


Aplikace pro zpětné testování akcií používá externí API pro získávání historických dat a fundamentálních ukazatelů. Hlavními API, která jsou v aplikaci použita, jsou SimFin pro fundamentální data a yfinance pro historické ceny akcií.

Pro získávání dat z API SimFin je použita funkce \texttt{load\_simfin\_data()}, která načítá roční finanční údaje jako příjmy, bilanční údaje, peněžní toky a historické ceny akcií. Tato data jsou následně uložena do databáze. Příklad kódu pro načítání dat vypadá takto:

\begin{lstlisting}[style=Python, caption= Simfin data] 

def load_simfin_data():
income_data = sf.load(dataset='income', variant='annual', market='us',
index=['Ticker', 'Fiscal Year'])
balance_data = sf.load(dataset='balance', variant='annual', market='us',
index=['Ticker','Fiscal Year'])
shareprices_data = sf.load(dataset='shareprices', variant='daily',
market='us', index=['Ticker', 'Date'])}
 \end{lstlisting}
Tato funkce načte data pro každý ticker (akcii) a uloží je do příslušných tabulek v databázi (např. \texttt{IncomeStatement}, \texttt{BalanceSheet}, \texttt{SharePrices}).

\section{Výpočet finančních ukazatelů}


Po získání dat jsou následně vypočítány klíčové finanční ukazatele, které jsou použity pro analýzu akcií a zpětné testování strategií. Výpočet těchto ukazatelů je prováděn v metodě \texttt{calculate\_ratios()}, která prochází všechny akcie a na základě dat v databázi vypočítá ukazatele jako P/E, ROA, ROE, a další.

Například výpočet P/E poměru probíhá následovně:

\begin{lstlisting}[style=Python, caption= Ratios] 
def calculate_ratios():
stocks = Stock.objects.all()
for stock in stocks:
income_statement = IncomeStatement.objects.filter(stock=stock).first()
if stock.market_cap and income_statement.net_income != 0:
stock.pe_ratio = round(stock.market_cap / income_statement.net_income, 2)}
\end{lstlisting}
Tato funkce pro každou akcii vypočítá P/E poměr na základě tržní kapitalizace a čistého zisku z posledního dostupného výkazu zisků a ztrát.

\section{Filtrace akcií podle kritérií}


Pro efektivní analýzu a zpětné testování je v aplikaci implementována filtrace akcií podle různých kritérií, jako jsou sektor, průmysl, a finanční ukazatele. Filtry jsou uloženy v uživatelských preferencích a aplikována na dotazy v databázi. Kód pro filtrování akcií podle vybraných parametrů vypadá takto:

\begin{lstlisting}[style=Python, caption= Filtr] 
filters = request.session.get('filters', [])
q_object = Q()
for f in filters:
field = f.get('field')
value = f.get('value')
q_object &= Q(**{f"field)
\end{lstlisting}

Tento filtr umožňuje uživatelům vyhledávat akcie na základě různých parametrů, jako jsou \texttt{sector} nebo \texttt{industry}, a následně je použít pro zpětné testování.

\section{Vizualizace a grafy}


Aplikace umožňuje vizualizaci historických dat akcií a výsledků zpětného testování. Pro tento účel je použitá knihovna \texttt{Plotly}, která generuje interaktivní grafy. Jeden z grafů, který zobrazuje vývoj ceny akcie, je generován takto:

\begin{lstlisting}[style=Python, caption= Grafy] 
<h2 class="subtitle">Graf vývoje ceny akcie</h2>
<div id="interactive-chart"></div>
<script src="https://cdn.plot.lyplotly-latest.min.js"></script>
<script>
const chartData = {{ chart_data|safe }};
Plotly.newPlot('interactive-chart', chartData.data, chartData.layout);
\end{lstlisting}

Tento kód využívá \texttt{Plotly} pro vykreslení interaktivního grafu, který zobrazuje data o vývoji ceny akcie na základě historických údajů. Grafy jsou dynamické a umožňují uživatelům interaktivně prozkoumávat data.



\chapter{Závěr a zhodnocení}


\section{Shrnutí výsledků}


Cílem této práce bylo vyvinout aplikaci pro zpětné testování akcií, která by uživatelům umožnila testovat jejich investiční strategie na historických datech a analyzovat výkonnost akcií na základě různých finančních ukazatelů. Po realizaci a testování aplikace lze shrnout následující výsledky:

\begin{itemize}
    \item Aplikace úspěšně integruje dvě klíčová externí API (SimFin a yfinance) pro získávání historických dat a finančních ukazatelů.
    \item Finanční ukazatele, jako jsou P/E, ROE, ROA, a další, jsou správně vypočítávány a uloženy do databáze pro následné použití v analýzách.
    \item Uživatelé mají možnost definovat vlastní investiční strategie a provádět zpětné testování na historických datech.
    \item Výsledky zpětného testování jsou generovány a vizualizovány pomocí interaktivních grafů, což zlepšuje analýzu a porozumění výsledkům.
    \item Aplikace umožňuje uživatelům sdílet portfolia.
\end{itemize}

Aplikace tedy splňuje své základní cíle a poskytuje funkční nástroj pro zpětné testování akcií.

\section{Hodnocení aplikace}


Aplikace byla vyvinuta s důrazem na uživatelskou přívětivost, bezpečnost a výkonnost. Testování aplikace ukázalo, že všechny základní funkce, jako je získávání dat z externích API, výpočet finančních ukazatelů a zpětné testování strategií, fungují správně. Dále byly provedeny testy na výkonnost aplikace, které potvrdily, že i při vyšším zatížení dokáže aplikace správně a efektivně zpracovávat velké objemy dat.

Jedním z hlavních přínosů aplikace je její flexibilita, která umožňuje uživatelům definovat vlastní investiční strategie a analyzovat je na historických datech. Výsledky zpětného testování jsou prezentovány v přehledné formě, která umožňuje uživatelům rychlou analýzu výkonnosti jejich strategií.


\section{Možnosti budoucího vývoje}


Vzhledem k tomu, že aplikace byla navržena s ohledem na rozšiřitelnost a flexibilitu, existuje několik možností pro její budoucí vylepšení. K těmto možnostem patří:

\begin{itemize}
    \item \textbf{Predikce pomocí AI a strojového učení} – Integrace modelů strojového učení pro predikci budoucího vývoje cen akcií na základě historických dat, klíčových finančních ukazatelů a dalších relevantních faktorů. To by uživatelům pomohlo lépe odhadnout potenciální výnosy a rizika.
    \item \textbf{Kalendář ekonomických událostí} – Přidání interaktivního kalendáře obsahujícího významné ekonomické události, jako jsou zveřejnění výsledků společností, makroekonomická data nebo rozhodnutí centrálních bank. Tato funkcionalita by uživatelům poskytla lepší přehled o možných rizicích a příležitostech.
    \item \textbf{Individuální zprávy pro každou akcii} – Zahrnutí sekce se souhrnem aktuálních zpráv a novinek souvisejících s každou akcií. Tímto způsobem by uživatelé mohli získat rychlý přehled o faktorech, které mohou ovlivnit cenu konkrétní akcie.
    \item \textbf{Vážení akcií v portfoliu} – Možnost vážit jednotlivé akcie v portfoliu podle uživatelem zadaných procent. To by umožnilo simulovat reálnější investiční strategie, kde mají různé akcie různou důležitost.
    \item \textbf{Pokročilé analýzy portfolia} – Zavedení nástrojů pro analýzu portfolia, jako je sledování celkové volatility, Sharpeho poměru nebo dalších ukazatelů, které by pomohly uživatelům optimalizovat jejich investice.
    \item \textbf{Sledování sentimentu trhu} – Možnost analyzovat sentiment trhu pomocí zpracování textů (např. z finančních článků a zpráv). Tato funkcionalita by uživatelům pomohla identifikovat, zda převládá pozitivní, neutrální nebo negativní nálada ohledně konkrétní akcie či trhu obecně.
    \item \textbf{Komunitní diskuse a hodnocení} – Integrované fórum nebo sekce pro komentáře by umožnily uživatelům diskutovat o investičních strategiích, ekonomických událostech nebo konkrétních akciích. Možnost hodnotit portfolia a strategie by dále podpořila interakci mezi uživateli.
\end{itemize}

Aplikace má také potenciál pro komerční využití, zejména pokud bude rozšířena o pokročilé funkce analýzy a optimalizace investičních strategií.

\section{Závěr}

Tato práce se zaměřila na vytvoření aplikace pro zpětné testování akcií, která by poskytovala nástroje pro analýzu akciových trhů na základě historických dat a finančních ukazatelů. Aplikace byla úspěšně navržena, implementována a otestována. Testování prokázalo správnost výpočtů a funkčnost aplikace i při velkém objemu dat.

Díky implementaci flexibilních analytických nástrojů a vizualizací má aplikace vysoký potenciál pro využití v praxi a pro rozšíření o nové funkce v budoucnu. Tento projekt tak představuje silný základ pro další vývoj nástrojů pro analýzu a testování akciových strategií.

 Zdrojový kód projektu je dostupný na GitHubu (https://github.com/Vojtik1/Alfint)

\chapter*{Seznam použitých informačních zdrojů}

\begin{itemize}
    \item \textbf{Django Documentation} – Django Software Foundation. Dostupné z: \url{https://www.djangoproject.com/}
    \item \textbf{yfinance Documentation} – Yahoo Finance API for Python. Dostupné z: \url{https://pypi.org/project/yfinance/}
    \item \textbf{SimFin API Documentation} – SimFin. Dostupné z: \url{https://simfin.com/docs}
    \item \textbf{The Intelligent Investor} – Benjamin Graham, HarperBusiness, 2003.
    \item \textbf{Plotly: A Comprehensive Guide} – Plotly Technologies Inc. Dostupné z: \url{https://plotly.com/python/}
    \item \textbf{ChatGPT} - ChatGPT. Dostupné z: \url{https://chatgpt.com/}
    \item \textbf{Stack Overflow} - Stack Overflow. Dostupné z: \url{https://stackoverflow.com/}
\end{itemize}




\end{document}
